\documentclass[11pt]{article}
\setlength{\topmargin}{-.5in}
\setlength{\textheight}{9in}
\setlength{\oddsidemargin}{.125in}
\setlength{\textwidth}{6.25in}
\usepackage{ae,aecompl}
\usepackage[T1]{fontenc}
\usepackage[utf8]{inputenc}
\usepackage{color}
\title{Simplifying method}
\begin{document}\maketitle
A Simplifying method is a type of method in an expression class used by the SimplifyAll function (in Data/EntityClass.py). It will try to return a simplified version of the class instance, but if it cannot/shouldn't simplify, a simplifying method must return False.
\\\\ For a simplifying method to be used in the SimplifyAll function, simply add the name of the method to the [NAME OF CLASS].simplifymethods file in Data/simplify methods. 

\end{document}