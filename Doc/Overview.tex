\documentclass[11pt]{article}
\setlength{\topmargin}{-.5in}
\setlength{\textheight}{9in}
\setlength{\oddsidemargin}{.125in}
\setlength{\textwidth}{6.25in}
\usepackage{ae,aecompl}
\usepackage[T1]{fontenc}
\usepackage[utf8]{inputenc}
\usepackage{color}
\usepackage{changepage}
\title{File overview}
\begin{document}\maketitle\noindent
This is an overview of the project
\section*{The root directory}
\begin{adjustwidth}{1em}{0pt}
\textbf{WebGUI.py} - The webserver that handles the Web Interface
\\\\\textbf{Sublime Text Files} - Integration with Sublime Text (look at instructions)
\\\\\textbf{Doc} - Documentation 
\\\begin{adjustwidth}{1em}{0pt}
\\\textbf{Tutorial} - Tutorial.cpt.pdf helps you get started using CasPyTex
\\\\\textbf{Data}
\\\begin{adjustwidth}{1em}{0pt}
	\textbf{Simplify Methods} - Holds the .simplifymethods files
	\\\\\textbf{Tests} - The .test files
	\\\\\textbf{Web Interface} - The Website for the Web Interface
	\\\\\textbf{CasPyTeXConfig.py} - The config - Uses python syntax
	\\\\\textbf{debugger.py} - A very simple debugging module
	\\\\\textbf{Entityclass.py} - The core of the CAS - holds the expression classes, as well as the Simplifying function
	\\\\\textbf{equationsolver.py} - Has the class equation, which is able to solve equations
	\\\\\textbf{latexfileclass.py} - Has the latexfile class, which can write a .tex and compile it to a .pdf
	\\\\\textbf{stringmanipulations.py} - Has some (not so) neat functions concerning strings
	\\\\\textbf{TextCAS.py} - The .cpt interpreter
	\\\\\textbf{textparser.py} - has the TextToCAS function, which converts strings to expression trees
	\\\\\textbf{unittester.py} - A script to execute the unit tests
\end{adjustwidth}
\end{adjustwidth}
\end{adjustwidth}
You might want to look at:
\begin{itemize}
\item Doc/Expression Class.pdf
\item Doc/Simplifying Methods.pdf
\item Doc/Tutorial/Tutorial.cpt.pdf
\end{itemize}
\end{document}